%-------------------------------------------------------------------------------
%	NAME:	report.tex
%	AUTHOR: Connor Beardsmore - 15504319
%	LAST MOD:	28/09/16
%	PURPOSE:	OOSE Assignment Report
%	REQUIRES:	NONE
%-------------------------------------------------------------------------------

\documentclass[]{article}
\usepackage[ margin=3cm ]{geometry}
\usepackage{graphicx}
\usepackage{fancyhdr}
\usepackage{float}
\usepackage{hyperref}
\usepackage{transparent}

\pagestyle{fancy}
\fancyhf{}
\lhead{Connor Beardsmore - 15504319}
\rhead{OOSE200}
\lfoot{October 2016}
\rfoot{\thepage}

\pagenumbering{arabic}
\graphicspath{{../images/}}

%-------------------------------------------------------------------------------
\begin{document}
%-------------------------------------------------------------------------------

\begin{titlepage}
	\begin{center}
		\vspace*{1cm}
		\LARGE\textbf{OOSE200 Report}
		\break
		Company Training Simulation
		\vspace{1cm}
		\break
		\Large\textbf{Connor Beardsmore - 15504319} 
		\vspace{2cm}
		\begin{figure}[H]
			\begin{center}
				{\transparent{0.7} 
					\includegraphics[height=0.4\textheight,width=0.7\textwidth]{placeholder.png}}
			\end{center}
		\end{figure}

		\vspace{4cm}
		\normalsize
		Curtin University \\
		Science and Engineering \\
		Perth, Australia \\
	    October 2016
	    
	\end{center}
\end{titlepage}

%-------------------------------------------------------------------------------

\vspace*{-0.8cm}
\begin{center}
	\section*{\textquotedblleft Company Training Simulation\textquotedblright}
\end{center}

%-------------------------------------------------------------------------------

\vspace*{0.8cm}
\section*{Polymorphism}

Throughout the Company Simulator, polymorphism is extensively utilized to both generalise and decouple code, leading to increased testability. To allow for the use of polymorphism, both implementation inheritance and interface inheritance has been employed.

\begin{itemize}
	\item Property - kept in map, polymorphically call calcProfit() via strategy
	\item Events + Plan - both use strategy so can call run() on parent class
	\item WageObserver list allows ANY class to become an observer if it implements
\end{itemize}

%-------------------------------------------------------------------------------

\section*{Design Pattern Implemented}

\subsubsection*{Factory Method Pattern}

A Factory was employed to encapsulate object instantiation for both the Event and Plan subclasses.

\subsubsection*{Dependency Injection Pattern}

The Dependency Injection pattern worked to remove all hard-coded dependencies, with the primary injector code being located in the main method.

\subsubsection*{Model View Controller Pattern}

The MVC pattern was utilized for the overall layout of the system, due to its flexibility and its strong separation of concerns.

\subsubsection*{Observer Pattern}

An observer was set up for WageEvents, allowing all relevant Property's to be updated easily by the notify method

\subsubsection*{Composite Pattern}

company owns other companies, profit tree structure

\subsubsection*{Template Method Pattern}

The Template Method pattern was used in the reading of the files. The common code for opening and closing files was kept in the superclass, with the subclasses implementing the protected abstract processLine() method.

\subsubsection*{Strategy Pattern}

The run() method located in both Event and Plan subclasses is a form of the strategy pattern, with each subclass implementing this method differently. Also utilizing the Strategy pattern is the calcProfit() method in Property subclasses, as all Properties calculate profit differently.

\subsubsection*{Miscellaneous Patterns}

The use of for each loops throughout the system illustrate a form of the simplistic Iterator pattern. The objects used for file reading utilize the Decorator pattern, however these classes were used from the Java API.

%-------------------------------------------------------------------------------

\section*{Testability}

\begin{itemize}
	\item Test cases!! sample outputs to clear up order ambiguity
	\item Factory  + Dependency Injection allow for easy mocking of objects, low coupling
	\item Mad toStrings() and debug output methods
	\item clear and consie exception handling
	\item tested on heaps of invalid file types for all 3 input files
\end{itemize}

%-------------------------------------------------------------------------------

\section*{Alternative Design Choices}

Despite the design having a high level of testability and maintainability, there are alternative design choices that could have been employed.

\
\begin{itemize}
	\item Iterators instead of for loops
	\item Controllers are easy to switch in and out, could have used one bunta controller
	\item could have used scanner instead of BufferedReader
\end{itemize}

%-------------------------------------------------------------------------------
\end{document}   
%-------------------------------------------------------------------------------